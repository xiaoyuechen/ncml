\documentclass[../proposal.tex]{subfiles}

\begin{document}

\section{SAT solver using genetic algorithms}
We propose to develop and evaluate a SAT solver which uses genetic algorithms.

\subsection{Introduction}
The satisfiability problem (SAT) is a NP-complete problem
(\cite{cook_1971}). Problems from Sudoku, N-queen to AI planning,
optimization and electric circuit design could all be modelled to SAT problems.
Hence, a SAT solver is a general purpose platform for solving various
real-world problems and has obvious importance.

\subsection{Background}
There are mainly two types of SAT solver technologies: complete methods and
incomplete methods. Complete methods can guarantee that it will eventually
report a satisfying assignment or prove not satisfiable, if it run for long
enough. On the other hand, incomplete methods has no such guarantee and
generally use stochastic local search (\cite{gomes_kautz_sabharwal_selman_2008}). In many cases
where the problem does not require a definitive answer, incomplete methods
could outperform complete methods in terms of both speed and memory usage.

Due to the heuristic nature of genetic algorithms, they could be used as a
comparison between 4 genetic SAT solvers and a current state-of-art local
search solver: WSAT. Their results show that evolutionary algorithms can
compete with WSAT. Some of them even exhibit better performance.

\subsection{Goal}
We would like to study the various aspects of a genetic SAT solver: the
representation of solution candidates, fitness functions, crossover, mutation,
and replacement strategies (\cite{gottlieb_marchiori_rossi_2002}). We would achieve that by
implementing a genetic SAT solver and evaluate it. We would also compare our
solver with a highly-tuned, SAT specific solver.

\subsection{Methods}
We would implement a genetic SAT solver in C/C++. Potentially, we would use
parallel computing libraries such as OpenMP or Cuda to exploit the parallel
search nature of genetic algorithms.

Data for evaluating will be generated using one of the existing SAT problem
generators. We will generate only satisfiable problems because they are
arguably the most meaningful for benchmarking.

As for the comparison, we would compare our solver with the existing solvers
using different benchmarking problems we generate. The rate of successfully
finding a satisfiable solution and the time taken will be important
measurements.

\end{document}
