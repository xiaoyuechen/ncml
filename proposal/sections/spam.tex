\documentclass[../proposal.tex]{subfiles}

\begin{document}

\section{Spam detection in Twitter}
 
\subsection{Introduction}
Online social networking sites are becoming leading communication methods for people and Twitter is one of the fastest growing one among them. However, the spammer also make use it as a tool to post some commercial advertises, harassing information or some other fraudulent information. The spam detecting job in Twitter is of necessity.

\subsection{Background}
There are some methods used to manually report spam by users, which require users to identify spam based on their own experience and they turn out to be inefficient and also caused many unnecessary problems. Therefore, lots of spam detection related surveys and studies has been done and plenty of them have made great results. The most common ones are the email spam detection and Web spam detection. For spam detection in Twitter, one machine learning approach is proposed in (Wang, 2010).

\subsection{Goal}
The goal is to apply methods of machine learning to automatically distinguish spam accounts from normal ones. And then we could do comparison between the algorithms. 

\subsection{Method}
We need to extract features from each Twitter user account for the purpose of spam detection. The features are extracted from different aspects which include graph-based features and content-based features, that is to analyses the metadata involving the set up of twitter accounts, and the other is to analyses text content and figure out if it’s a spam. To Facilitate the detection of spam,  three features based on graphic and three features based on content could be used. Three features based on on graphic are the number of friends, number of followers and the followers rate. Three features based on content are the number of duplicate tweets, the number of HTTP links, the number of replies(\cite{wang2010don}). For the data, one part is to extract user's detailed information from the Twitter's API for the graphic features mentioned above and the other is to extract a specific unauthorized user's recent tweets, which involves the web crawls. By comparing different classification algorithms like decision tree, neural network, support vector machines and Bayesian classification algorithm, we can get the most promising one and get some conclusions for the future work.

\end{document}
