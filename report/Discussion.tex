\section{Discussions}
In this section, we analyze the effects of different crossovers and the causes
of the performance difference between GNTSAT and WalkSAT. We will also point
out the limitations of this study.

\subsection{Different crossover operators}
In our experiments, we compared five different crossover methods. The results
show that different crossover operators have similar performance in terms of
both success rate and the average amount of computation required to produce
solutions.

On average, uniform crossover slightly outperform other types of crossovers in
most of the benchmark instances, in terms of both success rate and
computation. However, this would need to be further studied with more
benchmark instances and different parameters to confirm the statistical
significance. Hence we will not draw conclusions on the effects of different
crossovers until further studies are done.

\subsection{The search space}
\citeauthor{biere2009handbook} (\citeyear{biere2009handbook}) proposes a way to view the search space of
SAT problems as a \textit{landscape} in the space $\{0, 1\}^n \times \{0,1,\cdots,m\}$
where $n$ is the number of Boolean variables and
$m$ is the number of clauses. Each of the
$2^n$ assignments corresponds to a point in
$\{0,1\}$. Each point has a height ranging from
$0$ to $m$, which corresponds to the
number of clauses that are unsatisfied. Then the solutions to the SAT problem
are the points whose heights are $0$. Hence the search
problem for SAT is a search for the lowest points (global minima) in this
\textit{landscape}.

WalkSAT explores this landscape by randomly choosing a starting point and
descenting for $k$ iterations ($k$ is a
constant) using a greedy strategy with noises. If global minima are not found
within $k$ iterations, choose another random starting point
and repeat the process until a global minimum is found. The greedy strategy
with noises refers to committing ``greedy'' moves or ``random walk'' moves
according to some probability.

On the other hand, GNTSAT explores the search space by randomly choosing many
points as its population. It then selects parent points with probability
proportional to their heights in the landscape (lower points are more likely
to be selected), and uses crossover to generate children. The effects of
crossover could be viewed as generating new points in the search space. The
new points would be close to their parents' points if the parents are similar
to each other. Then the same greedy descents with noises are performed on the
children. The children will replace the oldest individuals in the population,
which corresponds to replacing the oldest points in the search space. This
procedure is repeated until a global minimum is found.

\subsection{Improvements of GNTSAT}
One limitation of the way WalkSAT explores the search space is that it
requires to start at a point from which a global minimum could be reached
within $k$ descenting operations. This requirement is very
hard to satisfy on difficult SAT instances where the ``distances'' between the
global minima and a random starting point are usually large. In these cases,
WalkSAT will have to restart for many times to search for a starting point
which is close enough to the global minimum. WalkSAT degrades to random search
when searching for the starting point. This is suboptimal if the search space
is huge.

An advantage of GNTSAT's exploration in the search space is that a point could
be used to create new points which are likely to be close to it. The new
points will then be descented using stochastic local search. This removes the
requirement that one of the starting points must be within
$k$ descenting operations from the global minimum. Note
that close children are only created when their parents are close. The parents
being close means there are multiple points converging to similar points, and
the heights of the points are among the lowest (otherwise the parents are not
likely to be selected). This in term implies GNTSAT would most likely continue
the local search on points which multiple other points converge to. This is
different from blindly increasing the amount of local search done on each
point.

GNTSAT practically make the original static max number of iterations
$k$ dynamic. $k$ would increase if more
points in the population appear in the same area, as these points have
increasing probability to be selected, and their children will be further
descented using stochastic local search. This is one of the major improvements
made by GNTSAT.

From the perspective of information, WalkSAT keeps very limited local
information: only the currently searched point. It discards the previous found
points entirely. GNTSAT keeps a population which contains the history of some
of the previously found bests. Instead of restarting the local search from a
random point, GNTSAT recombines the previously found bests to generate
children using crossover operators, and then use local search on the children.
Crossover operators utilize the information from the history and introduce
diversity. From this perspective, GNTSAT improves the utilization of computed
information.

\subsection{Limitations and further studies}
A major limitation is this study is the limited number of benchmark instances
and the number of repeated runs of each instance. We also only tested hard
uniform random instances. Hence the types of benchmarks need to be diversified
to include real-life problems.

A GNTSAT solver without any crossover operators should also be studied to
better understand the effects of crossover operators. The different crossover
operators should be tested on various benchmarks so that their effects are
differences could be studied.
