\section{Discussions}
In this section, we analyze the cause of the performance difference between
GNTSAT and WalkSAT.

\subsection{Different Crossover Methods}
In our experiments, we performed five different crossover methods. We could
find that uniform method got the best result in our experiments. It would not
surprise us because CC and FF are based on uniform crossover so they would do
more bit flips and take more time than uniform crossover. We have observed
that using CC and FF crossover would make it faster to converge on low
variable number benchmarks, and we believe that they would show more
advantages if we have more complex experiments. On the other hand, one-point
and two-point crossover are much more similar. These two kinds of methods are
doing consequent bit string swapping. However, this kind of swapping would
help little for our searching because clauses rarely include consequent bit
string. Thus, the only advance these two kinds of crossover giving us would be
maintaining population diversity.

\subsection{The search space}
\citeauthor{biere2009handbook} (\citeyear{biere2009handbook}) proposes a way to view the search space of
SAT problems as a \textit{landscape} in the space $\{0, 1\}^n \times \{0,1,\cdots,m\}$
where $n$ is the number of Boolean variables and
$m$ is the number of clauses. Each of the
$2^n$ assignments corresponds to a point in
$\{0,1\}$. Each point has a height ranging from
$0$ to $m$, which corresponds to the
number of clauses that are unsatisfied. Then the solutions to the SAT problem
are the points whose heights are $0$. Hence the search
problem for SAT is a search for the lowest points (global minima) in this
\textit{landscape}.

WalkSAT explores this landscape by randomly choosing a starting point and
descenting for $k$ iterations ($k$ is a
constant) using a greedy strategy with noises. If global minima are not found
within $k$ iterations, choose another random starting point
and repeat the process until a global minimum is found. The greedy strategy
with noises refers to committing ``greedy'' moves or ``random walk'' moves
according to some probability.

One limitaion of the way WalkSAT explores the search space is that it requires
to start at a point from which a global minimum could be reached within
$k$ descenting operations. This requirement is very hard to
satisfy on difficult SAT instances where the ``distances'' between the global
minima and a random starting point are usually large. In these cases, WalkSAT
will have to restart for many times to search for a starting point which is
close enough to the solution. WalkSAT degrades to random search when searching
for the starting point. This is suboptimal if the search space is huge.

On the other hand GNTSAT explores the search space by randomly choosing many
points as its population. It then selects parent points with probability
proportional to their heights (lower points are more likely to be selected),
and uses crossover to generate children. The effects of crossover could be
viewed as generating new points in the search space. The new points would be
close to their parents' points if the parents are similar to each other. Then
the same greedy descents with noises are performed on the children. The
children will replace the oldest individuals in the population, which
corresponds to replacing the oldest points in the search space with the new
points. This procedure is repeated until a global minimum is found.

An advantage of exploring the search space in this way is that a point could
be used to create new points which are likely be close to it. The new points
will then be descented using stochastic local search. This removes the
requirement that one of the starting points must be within
$k$ descenting operations from the global minimum. Note
that close children are only created when their parents are close. The parents
being close means there are multiple points converging to similar points, and
the heights of the points are among the lowest (otherwise the parents are not
likely to be selected). This in term implies GNTSAT would only continue the local
search on points that are converged towards from multiple other points, which saves
computation.

% \subsection{}
% Another limitation of WalkSAT is that it keeps very limited local states: only
% the currently searched point. It discards the previous found points entirely.
