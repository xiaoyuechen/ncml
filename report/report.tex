\documentclass[oneside,twocolumn,a4paper]{article}

% ========== Preamble (packages, definitions etc.) ==========

\usepackage[utf8]{inputenc}
\usepackage[english]{babel}
\usepackage{graphicx}
\usepackage{xcolor}
\usepackage{amsmath, amsthm, amssymb}
\usepackage{csquotes}
\usepackage{hyperref}
\usepackage{listings}
\usepackage{lmodern}
\usepackage{color}
\usepackage{subfiles}
%\usepackage[backend=bibtex,style=abbrv]{biblatex}
\newcommand{\todo}[1]{{\color{blue}#1}}  % show to-do items in blue
\setlength{\parskip}{\baselineskip}
\usepackage[activate={true,nocompatibility},
            final,
            tracking=true,
            kerning=true,
            spacing=nonfrench,
            factor=1100,
            stretch=10,
            shrink=10]{microtype}
\hypersetup{
    pdftitle={Project Report: Genetic Algorithm on SAT Problem},
    bookmarks=true,
    pdfpagemode=FullScreen,
}

\newcounter{questionnum} \setcounter{questionnum}{0}
%\newcommand{\question}[1]{%
%  \refstepcounter{questionnum}%
%  \paragraph{Question~\arabic{questionnum}:}{\emph{#1}}}

\newcommand\filltoend{\leavevmode{\unskip
  \leaders\hrule height.5ex depth\dimexpr-.5ex+0.4pt\hfill\hbox{}%
  \parfillskip=0pt\endgraf}}

\newcommand{\problem}[2]{%
	\vspace{-0.7em}
	\hspace{0.02\textwidth}
	\begin{minipage}[t][][b]{0.95\textwidth}
		{\bf \hspace{-0.015\textwidth}\makebox[7.5em][l]{{#1} ~~\filltoend}}%
		\hspace{1.2mm}{\it #2}%
	\end{minipage}
}

\def\email#1{{\tt#1}}

\lstset{ % Set the default style for code listings
	numbers=left,
	numberstyle=\scriptsize,
	numbersep=8pt,
	basicstyle=\scriptsize\ttfamily,
	keywordstyle=\color{blue},
	stringstyle=\color{red},
	commentstyle=\color{green!70!black},
	breaklines=true,
	frame=single,
	language=C,
	tabsize=4,
	showstringspaces=false
}

% ========== Title page ==========

\title{
	\includegraphics[width=0.6\textwidth]{UU_logo.pdf}\\[1em]
	Report for Natural Computation Methods for Machine Learning\\[1em]
	Project Report: Genetic Algorithm on SAT Problem\\[3em]
	Group 06
}

\author{
	Xiaoyue Chen \and
	Suling Xu \and
	Qinhan Hou
}

\begin{document}
\maketitle
\thispagestyle{empty} % Removes page number for front page
\clearpage
% ========== Document contents ==========
\abstract{
abstract
}
\newline
\newline
{\bf Keywords:} A, B, C

\subfile{Introduction}
\subfile{Background}
\subfile{Methods}
\subfile {Experiments}
\subfile {Discussion}
\subfile {Conclusion}


%\section{Questions}\noindent.
%
%\textbf{Question 1.}
%
%Write your answers here. You can cite like this.\cite{Smith:2012qr} The cite
%reference is in the file \texttt{bibliography.bib}. 
%
%\textbf{Question 2.}
%
%\textbf{Question 3.} 
%
%\textbf{Plot 1}
%\begin{figure}[h]
%    \begin{center}
%        \includegraphics[width=0.8\textwidth]{placeholder} % Include the image placeholder.png
%        \caption{Figure caption.}
%        \label{fig:my_figure}
%    \end{center}
%\end{figure}

% Delete this section if you have no plot to submit.
% Change the size of the figure by changing the value in [width=300pt]

% Putting \autoref{fig:my_figure} in your text will refer to the corresponding figure label.
% Eg.: "\autoref{fig:my_figure} clearly shows that the large circle is larger than the small box."
% Read more about autoref here https://en.wikibooks.org/wiki/LaTeX/Labels_and_Cross-referencing#autoref

%\bibliographystyle{apalike}
\bibliographystyle{abbrv}
\bibliography{bibliography}

\end{document}
