\section{Methods}
\subsection{The algorithm}
The population is initialized randomly.  In each iteration, first using
tournament to select some best individuals according to fitness. Then
crossover are performed on two individuals among the bests. Each generated
child is improved individually using local search, and then new individuals
are inserted to the population to replace the old individuals under some
conditions. Iterate the process until a solution is found. Our GNTSAT Solver
is described in detail as Algorithm \ref{alg:sat}.
\begin{algorithm}
	\SetAlgoLined
	\caption{SAT Solver Algorithm}
	\label{alg:sat}
	\KwData{population}
	\KwResult{a solution for the instance}
	\BlankLine
	\While{true}{
		\If{a solution is found}{
			return the solution\;
		}
		select some best individuals\; randomly choose two individuals among the bests
		for crossover\; \For{each each generated child}{
			local search\; \If{fitness (improved child) $>$ fitness (the oldest individual)}{
				insert the new child\; remove the oldest individual\;
			}
		}
	}
\end{algorithm}

\subsubsection{Representation}
An individual genome is a string of bits whose length equals to the number of
Boolean variables of the considered problem instance. Every Boolean variable
is associated to one bit. This is the most obvious way to encode a solution
candidate to a genome. It also makes mutation and crossover operations
computationally efficient. The most successful evolutionary algorithms for SAT
also uses representation \parencite{gottlieb_marchiori_rossi_2002}. Hence the bit string
representation is a reasonable starting point. A population is a set of such
individuals.

\subsubsection{Fitness function}
Let $x$ denote an individual ($x$ is a
bit string). The fitness value is equivalent to the number of satisfied
clauses of the individual.
\begin{equation*}
	\mathit{fitness}(x) = \sum_{i=1}^m c_i(x)
\end{equation*}
where $c_i(x)$ is the truth value of the
$i$th clause of individual $x$, and
$m$ is the number of clauses.

\subsubsection{Selection operation}
GNTSAT uses a modified version of tournament selection. First, a certain
number of individuals are randomly select to participate in the tournament.
The participating individuals are ranked according to fitness. The most fit
individuals will be selected and stored in a best individuals pool. The two
parents will be selected randomly from this pool.

\subsubsection{Crossover operators}
The main goal of the crossover operator in GNTSAT is to create potentially
promising new individuals. Using the selected parents, there are many ways for
them to crossover, and we consider 5 types: CC (Corrective Clause) Crossover,
F\&F (Fleurent and Ferland) Crossover, Uniform Crossover, One-point Crossover
and Two-point Crossover. We will introduce them respectively below.

\begin{itemize}
	\item
	      CC (Corrective Clause) Crossover: For each clause c that is unsatisfiable for
	      both parent X and parent Y solutions and for each variables that appears in
	      clause c, find the bit position using the literally absolute value
	      $i$ of the variable that produces the maximum improvement
	      on two parents guided by the improvement evaluation function. The function
	      equals to the number of false clauses which become true by flipping the
	      $ith$ bit in one of the solutions minus the number of
	      satisfied clauses which become false. Assign the flipped value applied to one
	      of the parents to the child in the bit position found, and all the bits with
	      no value of child take the value in corresponding position of parent X or
	      parent Y with the equal probability. CC crossover is illustrated in detail as
	      Algorithm \ref{alg:cc} \parencite{lardeux2006gasat}.

	      \begin{algorithm}
		      \SetAlgoLined
		      \KwData{two parents $X$ and $Y$}
		      \KwResult{one child $Z$}
		      \BlankLine
		      all the bits of child $Z$ take the values of parent
		      $X$ or parent $Y$ with the equal
		      probability\; \For{each clauses c such that $ \neg sat(X,c) \wedge \neg sat(Y,c)$}{
			      \For{all positions $i$ such that the variable
				      $x_{i}$ appears in c}{
				      Compute $ \sigma = improvement(X, i) + improvement(Y, i)$ \;
			      }{
				      Set $Z\vert k = flip(X\vert k)$ where $k$ is the position such
				      that $\sigma$ is maximum\;
			      }
		      } \caption{CC Crossover}
		      \label{alg:cc}
	      \end{algorithm}

	\item
	      F\&F (Fleurent and Ferland) Crossover: For each clause c that is satisfiable
	      for one parent and unsatisfiable for the other parent and for each variables
	      that appears in clause c, the bits positions in the child, corresponding to
	      the literally absolute value $i$ of the variable in
	      parents, are assigned values according to the parent satisfying the identified
	      clause. All the bits with no value of the child take the value in
	      corresponding position of parent X or parent Y with the equal probability. See
	      Algorithm \ref{alg:ff} \parencite{lardeux2006gasat}.

	      \begin{algorithm}
		      \SetAlgoLined
		      \caption{F\&F Crossover}
		      \label{alg:ff}
		      \KwData{two parents $X$ and $Y$}
		      \KwResult{one child $Z$}
		      \BlankLine
		      all the bits of child $Z$ take the values of parent
		      $X$ or parent $Y$ with the equal
		      probability\; \For{each clauses c such that $ sat(X,c) \wedge \neg sat(Y,c)$ (resp. $\neg sat(X,c) \wedge sat(Y,c))$}{
			      \For{all positions i such that the variable $x_{i}$ appears in c}{
				      $Z\vert i = X \vert i$ (resp. $Z\vert i = Y \vert i$)\;
			      }{
				      Set $Z\vert k = flip(X\vert k)$ where $k$ is the position such
				      that $\sigma$ is maximum\;
			      }
		      }
	      \end{algorithm}

	\item
	      Uniform Crossover: With equal probability, each bits of the child is chosen
	      from either parent. In that case, one new offspring is generated. See
	      Algorithm \ref{alg:uniform}.

	      \begin{algorithm}
		      \SetAlgoLined
		      \caption{Uniform Crossover}
		      \label{alg:uniform}
		      \KwData{two parents $X$ and $Y$}
		      \KwResult{one child $Z$}
		      \BlankLine
		      \For{each bit $x$}{
			      $Z\vert x = X\vert x$ or $Z\vert x = Y\vert x$ with equal possibility\;
		      }
	      \end{algorithm}

	\item One-point Crossover: Randomly select one pivot point (ranging between 0 to
	      length of the bits string) and exchange the substring from that bit point till
	      the end of the string between the two individuals. In that case, two new
	      offspring individual are generated. See Algorithm \ref{alg:onepoint}.

	      \begin{algorithm}[t]
		      \SetAlgoLined
		      \caption{One-point Crossover}
		      \label{alg:onepoint}
		      \KwData{two parents $X$ and $Y$ }
		      \KwResult{two children $Z_1$ and $Z_2$}
		      \BlankLine
		      \For{each bit $x$ where its position is less than the pivot
			      point}{
			      $Z_1\vert x = X\vert x$\;
			      $Z_2\vert x = Y\vert x$\;
		      }
		      \For{each bit x where its position is larger than the pivot point}{
			      $Z_1\vert x = Y\vert x$\;
			      $Z_2\vert x = Y\vert x$\;
		      }
	      \end{algorithm}

	\item
	      Two-point Crossover: Randomly pick two pivot points and the bits in between
	      the two points are swapped between the parent individuals. In that case, two
	      new offspring individual are generated. See Algorithm \ref{alg:twopoint}.

	      \begin{algorithm}[t]
		      \SetAlgoLined
		      \caption{Two-point Crossover}
		      \label{alg:twopoint}
		      \KwData{two parents $X$ and $Y$ }
		      \KwResult{two children $Z_1$ and $Z_2$}
		      \BlankLine
		      \For{each bit $x$ where its position is in between the two pivot
			      points}{
			      $Z_1\vert x = X\vert x$\;
			      $Z_2\vert x = Y\vert x$\;
		      }
		      \For{each bit $x$ where the position is not in between the two
			      pivot points}{
			      $Z_1\vert x = Y\vert x$\;
			      $Z_2\vert x = Y\vert x$\;
		      }
	      \end{algorithm}
\end{itemize}

\subsubsection{Mutation}
The WalkSAT local search algorithm is used as the mutation operator in GNTSAT
. WalkSAT is a randomized local search algorithm and illustrated in detail as
Algorithm \ref{alg:walksat} \parencite{biere2009handbook}. It attempts to
determine the best move by randomly choosing a clause among those that are
currently unsatisfied and selecting a variable to flip within it under some
conditions. The evaluation function in deciding bits to flip here is called
break-count. It equals to the number of satisfied clauses which become
unsatisfied by flipping a bit. The search stops when one solution is found.

\begin{algorithm*}[t]
	\SetAlgoLined
	\caption{WalkSat Local Search}
	\label{alg:walksat}
	\KwData{one child $Z$ generated from the crossover above }
	\KwResult{a solution for the instance}
	\BlankLine
	all the bits of child Z take the values of parent X or parent Y with the equal
	probability\; \For{ $i \leftarrow 1$ to MAX-Flips}{
		\If{$Z$ is a solution}{
			return $Z$\;
		}
		{Clause C is randomly selected among unsatisfied clauses\;}
		\eIf{$\exists$ variable $c \in C$ with
			$\operatorname{beak-count}=0$}{
			$v\leftarrow c$\;
		}{
			with probability p: $v\leftarrow$ a variable x in C chosen randomly\;
			with probability 1-p: $v\leftarrow$ a variable x in C with smallest
			$\operatorname{beak-count}$ \;
		}
		{Flip $v$ in Z\;}
	}
\end{algorithm*}

\subsection{Implementation}
GNTSAT is implemented in C++. GNTSAT takes a DIMACS CNF file as input and
output the resulting bit string if found. User could adjust parameters such as
population size, tournament size, and crossover operator, the maximum number
of flips in the local search subroutine, and the probability of a random move.
\footnote{Source code of GNTSAT is publicly available at \url{https://github.com/xiaoyuechen/ncml}}

\subsection{Performance measurements}
To make comparison with WalkSAT \parencite{selman1994noise} and to study the
effects of different crossovers operators, multiple runs are required for each
benchmark instance, and some measuring methods need to be taken for
statistically meaningful and relatively fair results. We consider three
measuring methods: SR (Success rate), AT (Average Time) and AFS (Average Flips
to Solution). We will introduce them respectively below.
\begin{itemize}
	\item
	      SR (Success rate): SR represents the percentage of runs where a solution has
	      been found within a time limit. Since some runs where the time to get a
	      solution is too long or the solver is stuck in the local optimal solution, we
	      set a maximum time for a successful run and use SR to measure the quality of
	      the solver.
	\item
	      AT (Average Time): AT represents the average seconds taken in successful runs.
	\item
	      AFS (Average Flips to Solution): Relating the computational costs to the
	      number of the basic moves in the search space for a solution has become the
	      standard measure used for studying the cost of SAT algorithms
	      \parencite{Singer2000}. AFS represents the average number of bit flips needed
	      to find a solution in successful runs, which was raised by
	      \citeauthor{Voss} and used to compare EAs generating new solution
	      candidates by single bit flips \parencite{Voss}.
\end{itemize}

Note that only successful runs' (runs where the solver could find a solution)
time and flip counts are taken to consideration when calculating the AT and
AFS.
