
\section{Experiments and results}
\subsection{Experiments}
\begin{figure*}[htbp]
	\centering
	\subfigure[AFS] {
		\begin{minipage}{7cm}
			\centering
			\includegraphics[width=6.5cm]{AFS}
		\end{minipage}
	}
	\subfigure[AT] {
		\begin{minipage}{7cm}
			\centering
			\includegraphics[width=6.5cm]{AT}
		\end{minipage}
	}
	\caption{The performance on datasets with different numbers of variables. We computed
		the average AFS and AT on all the 5 datasets with the same number of variables
		and then plot them in the figures.}
	\label{fig:performance}
\end{figure*}
We tested our algorithms in different benchmarks with variant clause numbers
to determine both its efficiency and effectiveness. We choose benchmarks with
100, 150, 200 and 250 variables, and 430, 645, 860 and 1075 clauses
respectively. The clause-variable ratio is fixed and equals to 4.3. The
benchmarks we used are all generated randomly. Datasets with 100, 150 and 200
variables are downloaded from SATLIB, an online published by
UBC\footnote{https://www.cs.ubc.ca/\textasciitilde hoos/SATLIB/benchm.html}. Datasets with 250 variables are generated by
ourselves. All of the benchmarks are satisfiable 3-SAT problems. In order to
reduce contingency, we run GNTSAT with each crossover operator and WalkSAT on
each benchmark instance for 10 times, and then we computed the average time
used and average number of flip to measure the efficiency. In addition, we had
set a two minutes timeout for the two algorithms, which means that failing to
find a solution within two minutes would be seem as a failure to solve the
instance. We would count the number of failures to determine the effectiveness
of the algorithms. The detailed results could be found in
Table.\ref{tab:results.}.

In our experiments we tested five kinds of crossover methods for genetic-based
algorithm, they were CC, F\&F, uniform, one-point and two-point crossover
respectively. Besides we ran WalkSAT on the same dataset as the contrast to
find the difference between our algorithms and local search methods.

\subsection{Results}

Table.\ref{tab:results.} shows that as the number of variables increase,
the AFS and AT of all the algorithms raised, but the SR decreased. For
datasets with 100 and 150 variables, the SRs of all the algorithms were 100\%,
but algorithms began to fail to solve problems when the number of variables
were increased to 200 (for example, dataset UF-200-068). From
Fig.\ref{fig:performance} it could be found that there were only slight
differences between different methods while the number of variables was low.
However, while the problems becoming more complex, the difference would show
up. For those datasets having 250 variables, the WalkSAT algorithm would take
a long time to search the solution, and has less success rate.

Among the 50 runs of the UF-250 benchmark instance set, GNTSAT with uniform
crossover found solutions for 46 times, while WalkSAT found solutions for 23
times. GNTSAT with other types of crossovers also showed similar results. The
success rate of GNTSAT is significantly higher than WalkSAT.

Overall, GNTSAT using uniform crossover performed the best on the complex
3-SAT problems, both in terms of computation and success rate. For other types
of crossover operators, CC, one-point, and two-point crossover performed very
similar, while F\&F had slightly better performance.
