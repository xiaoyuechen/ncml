
\section{Experiments \& Results}
\subsection{Experiments}
\begin{figure*}[htbp]
	\centering
	\subfigure[AFS] {
		\begin{minipage}{7cm}
			\centering
			\includegraphics[width=6.5cm]{AFS}
		\end{minipage}
	}
	\subfigure[AT] {
		\begin{minipage}{7cm}
			\centering
			\includegraphics[width=6.5cm]{AT}
		\end{minipage}
	}
	\caption{The performance on datasets with different numbers of variables. We computed
		the average AFS and AT on all the 5 datasets with the same number of variables
		and then plot them in the figures.}
	\label{fig:performance}
\end{figure*}
Here we performed our algorithms in different datasets with variant clause
numbers to determine both its efficiency and effectiveness. We choose datasets
with 100, 150, 200 and 250 variables, corresponding to 430, 645, 860 and 1075
clauses respectively. That is to say, the rates of a number of variables and
clauses are fixed and equals to 4.3. We chose ratio 4.3 because it gives computationally challenging instances when the instances are of 3-SAT \parencite{BenchmarksRatio}. The datasets we used are all generated
randomly. Datasets with 100, 150 and 200 variables are downloaded from SATLIB,
an online published by UBC\footnote{https://www.cs.ubc.ca/\textasciitilde hoos/SATLIB/benchm.html}. Datasets with 250 variables
are generated by ourselves. All of the datasets are satisfiable for 3-SAT
problems. In order to reduce contingency, we run the genetic algorithm-based
method and the WalkSAT method in each dataset 10 times, and then we computed
the average time used and average flip times to measure the efficiency. In
addition, we had set a two minutes timeout for the two algorithms, which means
that failing to find an available solution in two minutes would seem a failure
to solve the dataset. We would count the number of failures to determine the
effectiveness of the algorithms. The detailed results could be found in
Table.\ref{tab:results.}.

In our experiments we tested five kinds of crossover methods for genetic-based
algorithm, they were CC, F\&F, uniform, one-point and two-point crossover
respectively. Besides we run WalkSAT on the same dataset as the contrast to
find the difference between our algorithms and local search methods.

\subsection{Results}

According to Table.\ref{tab:results.}, it is obvious that with the
variables increasing, the AFS and AT of all the algorithms raised, but the SR
decreased. For datasets with 100 and 150 variables, the SRs of all the
algorithms were 100\%, but algorithms began to fail solving problems when it
came to 200 variables (for example, dataset UF-200-068). From
Fig.\ref{fig:performance} it could be found that there were only slight
differences between different methods while the number of variables was low.
However, while the problems becoming more complex, the difference would show
up. For those datasets having 250 variables, the WalkSAT algorithm would take
a long time to search the solution, and there would be a high possibility that
it could not give a solution. Overall, the genetic-based algorithm using
uniform crossover performed best on the complex 3-SAT problems, both in
computation complexity and time consumption aspects. For other kinds of
crossover methods, CC, one-point, and two-point crossover performed most the
same, while F\&F had shown a little superiority over them.
