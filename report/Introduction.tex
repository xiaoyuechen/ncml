\section {Introduction}
The satisfiability problem (SAT) is a NP-complete problem
\parencite{cook_1971}. SAT problem consists a set of Boolean variables
$x_1, \cdots, x_n$ and a Boolean formula $f: \mathbb{B}^n \rightarrow \mathbb{B}=\{0, 1\}$. The
question is whether an assignment $x=(x_1, \cdots, x_n)$ exists such that
$f(x)=1$ \parencite{gottlieb_marchiori_rossi_2002}. A wide range of problems can be
expressed as SAT problems. This includes verification, planning, scheduling
and combinatorial design \parencite{biere2009handbook}. Hence, a SAT solver has been a
general purpose platform for solving various real-world problems and has
obvious importance.

There are two types of SAT solver technologies: complete methods and
incomplete methods. Complete methods can guarantee that it will eventually
report a satisfying assignment or prove not satisfiable, if it run for long
enough. Complete methods have an exponential time complexity, unless P = NP.
On the other hand, incomplete methods has no such guarantee and generally use
stochastic local search \parencite{gomes_kautz_sabharwal_selman_2008}. In many cases where the problem
does not require a definitive answer, incomplete methods could outperform
complete methods in terms of both speed and memory usage.

Genetic algorithms (GAs) are incomplete methods that have been applied to many
NP-complete problems, including SAT \parencite{gottlieb_marchiori_rossi_2002}. However, results
suggest that classical GAs may not outperform local search algorithms
\parencite{de1989using}. Nevertheless, recent results show that GAs can yield
good results if combined with other methods \parencite{gottlieb_marchiori_rossi_2002}.

This paper proposes a genetic algorithm---GNTSAT, which is based on WalkSat
\parencite{selman1994noise} and outperforms WalkSat in many benchmark instances.
This paper introduces the GNTSAT algorithm, makes performance evaluations
based on benchmarks, and discuss the possible reasons for the performance
difference and their implications.
