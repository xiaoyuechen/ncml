\section{Background}
Incomplete methods for SAT can significantly outperform complete methods in
many domains. Hence there has been a tremendous amount of research done on them
since the early 1990s. SAT problems can be effectively solved by using greedy
decent if they do not have any local minima, as it could rapidly approach the
global minima without exhaustively branching or backtracking. However, most
interesting problems do have local minima. Hence the main challenge and
opportunity for incomplete SAT methods is escaping local minima
\parencite{biere2009handbook}.

WalkSAT \parencite{selman1994noise} plays a key role in the success of local search
for satisfiability \parencite{biere2009handbook}. It improves its ability to escape
local minima by introducing noise into search in the form of uphill moves. It
always selects a variable to flip from a randomly chosen unsatisfied clause
$C$. If flipping a bit in $C$  does not
turn any currently satisfied clauses to unsatisfied, WalkSAT flips the bit (a
``freebie'' move). Otherwise, according to a change, it either makes a
``greedy'' move (a bit flip that will minimize the number of currently
satisfied clauses becoming unsatisfied) or a ``random walk'' move (flipping a
random bit in $C$). WalkSAT is proved to significantly
outperform greedy descent methods on hard satisfiable problems. GNTSAT uses
WalkSAT on generated children as its mutation operator.


FlipGA \parencite{marchiori1999flipping} is a genetic local search algorithm. FlipGA uses
a small population size (10 in the original paper). It selects parents with
probability proportional to their fitness and generates children with uniform
crossover. Its mutation operator has a 0.9 probability to be used on each
individual, with a 0.5 probability to flip each gene. FlipGA then performs a
greedy local search on the children. It scans the genes in random order and
tries to flip each one of them. If a flip result in an increase of satisfied
clauses, the flip is accepted. This process is repeated if the obtained
chromosome has more satisfied clauses after all the genes have been
considered. Benchmark results from \cite{gottlieb_marchiori_rossi_2002} show that FlipGA has
a similar performance to WalkSAT.

GASAT \parencite{lardeux2006gasat} is another genetic local search algorithm. It
combines many local search methods and is more complex than the previously
mentioned algorithms. GASAT uses a population scheme where the newly created
child is accepted only if it is fitter than a subset of the population. The
child is inserted into the population to replace the oldest individual in the
population. \citeauthor{lardeux2006gasat} experimented with different crossover
operators and remarked that Corrective Clause crossover (CC) and Corrective
Clause and Truth Maintenance crossover (CCTM) have better behaviors than other
types of crossovers in the sense that they could quickly reduce the number of
false clauses while maintaining a higher level of population diversity than
other crossovers. GASAT uses tabu search (TS) to help avoid local minima. It
also uses TS with diversification to solve the \textit{stumble clause} problem
i.e., the last false clause is often the same clause and blocks to tabu search.

\citeauthor{lardeux2006gasat} (\citeyear{lardeux2006gasat}) made a comprehensive comparison between GASAT and other
algorithms, including WalkSAT and FlipGA. The results show that GASAT has
reliable and competitive performance. Inspired by GASAT, GNTSAT uses a similar
population management scheme to GASAT. We will also make comparisons on
different crossover operators mentioned by \citeauthor{lardeux2006gasat}.
