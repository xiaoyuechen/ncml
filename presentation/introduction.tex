\section{Introduction}

\begin{frame}
	\frametitle{An Interesting Scenario}
	\begin{columns}
		\begin{column}{.4\textwidth}
		\begin{figure}
			\centering
			\includegraphics[width=5cm]{Mumin}
		\end{figure}
		\end{column}
		\begin{column}{.5\textwidth}
		\begin{alertblock}{Mumin and his friends}
			After vaccinated, Mumin and his friends want to plan a tour over Europe...
			\begin{itemize}
				\item Mumin had gone to Berlin and wants to visit Paris and Madrid.
				\item Snufkin wants to visit London and Berlin but had gone to Madrid.
				\item Sniff wants to visit Madrid and London, but had gone to Paris.
			\end{itemize}
		So how to plan the tour?
		\end{alertblock}	
		\end{column}
	\end{columns}	
\end{frame}

\begin{frame}
	\frametitle{Incomplete Methods for SAT problems}
	\begin{equation*}
	\scriptsize
		\overbrace{(Paris \vee \neg Berlin \vee Madrid)}^{Mumin} \wedge \overbrace{(\neg Madrid \vee London \vee Berlin)}^{Snufkin} \wedge \overbrace{(\neg Paris \vee London \vee Madrid)}^{Sniff}
	\end{equation*}
	\large	
	\begin{alertblock}{What if we choosing another way?}
		Using incomplete methods is kind of taking a step back.
		\begin{itemize}
	\normalsize
		\item If there are numerous variables, then using complete methods is hard to get a solution.
		\item Sometimes we do not need to prove if a proposition is satisfiable.
		\item Many incomplete methods have been proved successfully solving SAT problems\parencite{biere2009handbook}.
	\end{itemize}
	\end{alertblock}
	
	
\end{frame}

\begin{frame}
	\frametitle{Why Genetic Algorithm?}
	\Large
	\begin{itemize}
		\item Genetic algorithm is a heuristic searching method that widely used(\citeauthor{Voss}).
		\item Numerous incomplete solvers have been invented to solve SAT problems \parencite{marchiori1999flipping, selman1994noise} but genetic algorithm was rarely considered.
		\item The main challenge for incomplete methods is how to escape local minima.
	\end{itemize}
	
\end{frame}