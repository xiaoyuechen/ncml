\section{Introduction}

\begin{frame}
	\frametitle{An Interesting Scenario}
	\begin{columns}
		\begin{column}{.4\textwidth}
		\begin{figure}
			\centering
			\includegraphics[width=5cm]{Mumin}
		\end{figure}
		\end{column}
		\begin{column}{.5\textwidth}
		\begin{alertblock}{Mumin and his friends}
			If you want to invite Mumin and his friends to a party...
			\begin{itemize}
				\item Mumin is busy at Friday but free at Monday and Tuesday.
				\item Snufkin is free at Tuesday but busy at Monday and Wednesday.
				\item Sniff is busy at Friday, Sunday and Thursday.
			\end{itemize}
		So which day would you like to schedule your party? 
		\end{alertblock}	
		\end{column}
	\end{columns}	
\end{frame}

\begin{frame}
	\frametitle{Incomplete Methods for SAT problems}
	\begin{equation*}
	\small
		\overbrace{(Mon. \vee \neg Fri. \vee Tues.)}^{Mumin} \wedge \overbrace{(\neg Mon. \vee \neg Wed. \vee Tues.)}^{Snufkin} \wedge \overbrace{(\neg Fri. \vee \neg Sun. \vee \neg Thur.)}^{Sniff}
	\end{equation*}
	\large	
	\begin{alertblock}{What if we choosing another way?}
		Using incomplete methods is kind of taking a step back.
		\begin{itemize}
	\normalsize
		\item If there are numerous variables, then using complete methods is hard to get a solution.
		\item Sometimes we do not need to prove if a proposition is satisfiable.
		\item Many incomplete methods have been proved successfully solving SAT problems.
	\end{itemize}
	\end{alertblock}
	
	
\end{frame}

\begin{frame}
	\frametitle{Why Genetic Algorithm?}
	\Large
	\begin{itemize}
		\item Genetic algorithm is a heuristic searching method that widely used.
		\item Numerous incomplete solvers have been invented to solve SAT problems but genetic algorithm was rarely used.
		\item The main challenge for incomplete methods is how to escape local minima.
	\end{itemize}
	
\end{frame}