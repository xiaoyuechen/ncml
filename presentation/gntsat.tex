\section{GNTSAT}

\begin{frame}
	\frametitle{GNTSAT}
\begin{small}
	\renewcommand{\thealgocf}{}
	\begin{algorithm}[H]
		\SetAlgoLined
		\KwData{randomly generated population}
		\KwResult{a solution for the instance}
		\BlankLine
		\While{true}{
			\If{a solution is found}{
				return the solution\;
			}
			select some best individuals\; randomly choose two individuals among the bests
			for crossover\; \For{each generated child}{
				local search\; \If{fitness (improved child) $>$ fitness (the oldest individual)}{
					replace the oldest individual with the new child\;
				}
			}
		}
		\caption{GNTSAT solver}
	\end{algorithm}
\end{small}
\end{frame}


\begin{frame}
	\frametitle{GNTSAT}
	\begin{block}{Selection}
		Use a modified version of tournament selection to get healthy parents for crossover.
	\end{block}
	\begin{block}{Crossover}
	We consider 5 types of crossover to create potentially promising new individuals: Corrective Clause (CC) crossover,
	Fleurent and Ferland (F\&F) crossover, uniform crossover, one-point crossover. By analyzing the results of applying different crossovers, we could get the best one.
	\end{block}
	\begin{block}{Mutation}
		Use WalkSAT local search that interleaves the greedy moves with random walk moves, trying to determine the best move by flipping a variable.
	\end{block}
\end{frame}
\begin{frame}[t]
	\frametitle{GNTSAT}
	\framesubtitle{Differences between traditional genetic SAT solver and GNTSAT}
	\begin{block}{Difference in population iteration}
		\begin{columns}[t]
			\begin{column}{0.5\textwidth}
				\textbf{Traditional genetic SAT solver:}
				\begin{itemize}
				\item perform genetic operators on each individual accordingly
				\item update the whole generation with new generation
				\end{itemize}
			\end{column}
			\begin{column}{0.5\textwidth}
				\textbf{GNTSAT:}
				\begin{itemize}
				\item choose a pair of parents among the bests to crossover and then mutate the child. 
				\item replace the olderest individual with the newly generated child.
				\end{itemize}
				\end{column}
			\end{columns}
	\end{block}
\end{frame}

\begin{frame}[t]
	\frametitle{GNTSAT}
	\framesubtitle{Differences between traditional genetic SAT solver and GNTSAT}
	\begin{block}{Difference in mutation operator}
		\begin{columns}[t]
			\begin{column}{0.5\textwidth}
				\textbf{Traditional genetic SAT solver:}
				\begin{itemize}
				\item flip one or more variables in a candidate solution from its initial state
				\end{itemize}
			\end{column}
			\begin{column}{0.5\textwidth}
				\textbf{GNTSAT:}
				\begin{itemize}
				\item use the WalkSAT local search as the mutation operator.
				\item always start with a newly generated child that is near to the solution.
				\item randomly pick a clause among those that are currently unsatisfied, then flip a variable within that clause with some conditions.
				\end{itemize}
				\end{column}
			\end{columns}
	\end{block}
\end{frame}
